\documentclass[a4paper,12pt,titlepage]{book}


% **************************************************
% ******************* Usepackages ******************
% **************************************************

%\pagestyle{plain}
\usepackage[utf8]{inputenc}
\usepackage[ngerman]{babel}
\usepackage{float} %Bilder Positionierung
\usepackage{amsmath} % AMS Math Package
\usepackage{amsthm} % Theorem Formatting
\usepackage{amssymb} % Math symbols such as \mathbb
\usepackage{mathtools} %tools formathematical typesetting, based on amsmath
\usepackage{dsfont}
\usepackage{trsym}
\usepackage{pifont}
\usepackage{tabularx}
\newcolumntype{L}[1]{>{\raggedright\arraybackslash}p{#1}}
\newcolumntype{C}[1]{>{\centering\arraybackslash}p{#1}}
\newcolumntype{R}[1]{>{\raggedleft\arraybackslash}p{#1}}
%\usepackage[ngerman]{refstyle}
%\usepackage{empheq}
%\usepackage{units} %zum Anzeigen der Einheiten
\usepackage{nicefrac} %schöne Brüche in Zeile, in units inkludiert?!
%\usepackage{braket} %BraKet-Notation
\usepackage{fix-cm}  %erlaubt schriftgröße beliebig zu verändern
\usepackage{icomma} %keine Lehrzeichen bei Fließkommazahlen
\usepackage{colortbl} %Tabellen einfaerben
\usepackage{booktabs} %"'schöne"' Tabellen ?
\usepackage{arydshln} %kein Darstellungsfehler in Tabelle
\usepackage[FIGTOPCAP]{subfigure} %teilt figures, 2Bilder nebeneinander möglich
%\setlength{\subfigcapskip}{0pt}

\usepackage[colorlinks=true,linkcolor=black,citecolor=black, pdfsubject={Diplomarbeit: Aufbau und Charakterisierung einer interferometrischen Frequenzstabilisierung für Diodenlaser}, pdfproducer={Thomas Fischbach},
pdfcreator={Thomas Fischbach},
pdfauthor={Thomas Fischbach},
plainpages=false,
pdfstartview=Fit,
pdfpagelayout=TwoPageRight]{hyperref} %interne Verlinkung / Metadaten
\usepackage[font=footnotesize,
labelfont=bf,
%margin=18pt,
%textfont=sl,
%labelsep=endash,
%tableposition=top
]{caption}
\usepackage{tikz} %zum Zeichnen von grafiken
\usetikzlibrary{arrows,decorations.pathmorphing,decorations.shapes,shapes,backgrounds,positioning,fit,petri,calc}

\usepackage{palatino}
\usepackage{listings}
\usepackage[T1]{fontenc}
\usepackage{babel}
%\usepackage[style=verbose-trad2,natbib=true]{biblatex}
%\usepackage[babel,german=quotes]{csquotes}
%\usepackage{cite}
\usepackage[numbers,square]{natbib}

%\usepackage{graphicx} % Allows for eps images
\usepackage{multicol} % Allows for multiple columns
%\usepackage[Lenny]{sty/fncychap} %Kapitel Style
%\usepackage{color,calc,soul,fourier}
\usepackage{xcolor}
\usepackage{blindtext}
\usepackage{textcomp}
\usepackage{framed}
\usepackage{lscape}

%Code einbetten
\usepackage{courier}
\usepackage{listings}
\lstset{numbers=left, numberstyle=\tiny, numbersep=5pt}
\lstset{language=C}
\lstset{basicstyle = \ttfamily\small\mdseries}



% **************************************************
% ******************* Seitenformatierung ***********
% **************************************************
% DINA4 210x297mm   1in=25,4mm

\setlength{\hoffset}{-20mm}            %ca -42mmm ist am linken Rand
\setlength{\voffset}{-5mm}
\setlength{\topmargin}{10mm}
\setlength{\textwidth}{150mm}
\setlength{\textheight}{240mm}
\setlength{\marginparwidth}{0mm}
\setlength{\marginparsep}{0mm}
\setlength{\topmargin}{0mm}

\setlength{\oddsidemargin}{30mm}

\frenchspacing %Disabled extra-Leerzeichen nach Satzende

\captionsetup{font=footnotesize,format=plain,indention=0.5cm,labelfont=bf}


% **************************************************
% ******************* Dokumentdaten ****************
% **************************************************

\title{Aufbau und Charakterisierung einer interferometrischen Frequenzstabilisierung für Diodenlaser}
\author{Diplomarbeit\\von\\Thomas Fischbach}
\date{\today}


% **************************************************
% ******************* New Commands *****************
% **************************************************

\newcommand{\VS}{\vspace{5mm}}
\newcommand{\VNI}{\vspace{5mm} \noindent}
\newcommand{\cm}{\mbox{cm}^{-1}}

\newcommand{\tikzgrid}{\draw[thick] (-5,0) -- (5,0);
                \draw[thick] (-3,2) -- (3,2);
                \draw[thick] (-3,-2) -- (3,-2);
                \draw[thick] (0,-5) -- (0,5);
                \draw[thick] (2,-3) -- (2,3);
                \draw[thick] (-2,-3) -- (-2,3);}
               
\newcommand{\graph}{11.0cm}
\input{jkcommands}
\usepackage{titlesec}
\usepackage{ifpdf}

\graphicspath{{gfx/}{plt/}} 


% **************************************************
% **************** eps in pdflatex******************
% **************************************************

\newif\ifpdf
\ifx\pdfoutput\undefined
   \pdffalse
\else
   \pdfoutput=1
   \pdftrue
\fi
\ifpdf
   \usepackage{graphicx}
   \usepackage{epstopdf}
   \epstopdfsetup{suffix=}
   \DeclareGraphicsRule{.eps}{pdf}{.pdf}{`epstopdf #1}
   \pdfcompresslevel=9
\else
   \usepackage{graphicx}
\fi


% **************************************************
% **************** Chapter style *******************
% **************************************************


%style 1

% \titleformat{\chapter}[display]
% {\bfseries\Large}
% {\hfill \tikz[remember picture] \node[] (nr)
% {\fontsize{50}{70}\selectfont\color{red!50!black}\textbf{\thechapter}~};
% \begin{tikzpicture}[overlay,remember picture] \coordinate (leftborder) at ($(nr)-(100,0)$);
% \coordinate (leftborderhigh) at ($(nr.north west)-(100,0)$);
% \coordinate (leftborderlow) at ($(nr.south west)-(100,0)$);
% \coordinate (left) at ($(nr.west)-(1.5,0)$);
% \fill[red!20!white] (leftborderhigh) -- ($(nr.north west)+(0,0.5)$) -- (left)
% -- (leftborder) -- cycle; \fill[red!20!white] (leftborderlow) -- ($(nr.south
% west)-(0,0.5)$) -- (left) -- (leftborder) -- cycle; \draw[red!50!black,line width=0.1em,line join=round] (left) -- ($(nr.south
% west)-(0,0.5)$) -- ($(nr.south east)-(0,0.5)$) -- ($(nr.north east)+(0,0.5)$) -- ($(nr.north west)+(0,0.5)$) -- (left) -- (leftborder);
% \end{tikzpicture}}
% {-2ex}
% {\filleft\fontsize{50}{70}\selectfont\scshape}
% [\vspace{0ex}]


%style 2

% \colorlet{chapter}{black!75}
% \addtokomafont{chapter}{\color{chapter}}
% 
% 
% %\makeatletter% siehe De-TeX-FAQ
% \renewcommand*{\chapterformat}{%
% \begingroup% damit \unitlength-Änderung lokal bleibt
% \setlength{\unitlength}{1mm}%
% \begin{picture}(20,40)(0,5)%
% \setlength{\fboxsep}{0pt}%
% %\put(0,0){\framebox(20,30){}}%
% %\put(0,20){\makebox(20,20){\rule{20\unitlength}{20\unitlength}}}%
% \put(20,15){\line(1,0){\dimexpr
% \textwidth-20\unitlength\relax\@gobble}}%
% \put(0,0){\makebox(20,20)[r]{%
% \fontsize{28\unitlength}{28\unitlength}\selectfont\thechapter
% \kern-.04em% Ziffer in der Zeichenzelle nach rechts schieben
% }}%
% \put(20,15){\makebox(\dimexpr
% \textwidth-20\unitlength\relax\@gobble,\ht\strutbox\@gobble)[l]{%
% \ \normalsize\color{black}\chapapp~\thechapter\autodot
% }}%
% \end{picture} % <-- Leerzeichen ist hier beabsichtigt!
% \endgroup
% }


%style 3

% \definecolor{nicered}{rgb}{.647,.129,.149}
% \usepackage{soul}
%  
% \makeatletter
% \newsavebox{\feline@chapter}
% \newcommand\feline@chapter@marker[1][4cm]{%
% \sbox\feline@chapter{%
% \resizebox{!}{#1}{\fboxsep=1pt%
% \colorbox{nicered}{\color{white}\bfseries\sffamily\thechapter}%
% }}%
% \rotatebox{90}{%
% \resizebox{%
% \heightof{\usebox{\feline@chapter}}+\depthof{\usebox{\feline@chapter}}}%
% {!}{\scshape\so\@chapapp}}\quad%
% \raisebox{\depthof{\usebox{\feline@chapter}}}{\usebox{\feline@chapter}}%
% }
% \newcommand\feline@chm[1][4cm]{%
% \sbox\feline@chapter{\feline@chapter@marker[#1]}%
% \makebox[0pt][l]{% aka \rlap
% \makebox[1cm][r]{\usebox\feline@chapter}%
% }}   
%  
% \renewcommand*{\chapterformat}{%
% \hspace{\leftmargin} \feline@chm[2.5cm] % Height of the colored box
% \hspace{2cm}
% }
% \makeatother

%style 4

% \titleformat{\chapter}[display]
% {\bfseries\Large}
% {\hfill \tikz[remember picture] \node[] (nr) {\fontsize{120}{70}\selectfont\color{lightgray}\textbf{\thechapter}};
% \begin{tikzpicture}[overlay,remember picture]
% \coordinate (leftborder) at ($(nr)-(100,0)$);
% \coordinate (left) at ($(nr.west)-(1.5,0)$);
% \draw[decoration={shape backgrounds,shape size=.5cm,shape=signal},signal from=west, signal to=east,decorate, draw=red!50!black, fill=red!50, decoration={shape sep=.5cm},line join=round] (leftborder) -- (left);
% \end{tikzpicture}}
% {-2ex}
% {\filleft\fontsize{50}{70}\selectfont\scshape}
% [\vspace{0ex}]
