In diesem Kapitel werden die Messergebnisse vorgestellt, die eine
Charakterisierung des Lasersystems erlauben. Dabei werden zunächst Messungen
aufgeführt, die die reine Frequenzstabilität der Laser beschreiben. Dazu werden
Lang- und Kurzzeitverhalten sowohl des alten als auch den neuen Lasersystems
untersucht (Abschn. \ref{sec:laserstabilitaet}). Weiterhin wird das
Linearitätsverhalten der \textit{iScans} charakterisiert, wobei die
Nichtlinearitäten direkt und deren Auswirkung auf die
Frequenzverstimmungsroutine gemessen werden (Abschn.
\ref{sec:linearisierung}). Abschnitt \ref{spektroskopie_an_uran} beschäftigt
sich mit den bishereigen spektroskopischen Messungen an Uran mit dem alten und
neuen System, dem bisher verwendeten Anregungsschema und der Suche nach neuen
Anregungsschemata. In Abschn. \ref{sec:countraten_fluktuation} wird kurz auf die
Gesamtstabilität der Systeme eingegangen und Countratenfluktuationen vorgstellt.
Zu Beginn dieses Kapitels soll allerdings noch eine Charakterisierung des
verwendeten FPIs besprochen werden.

\section{Charakterisierung des FPIs}\label{charakterisierung_FPI}
Um korrekte Relativfrequenzen berechnen und anfahren zu können ist von
essenzieller Wichtigkeit, den FSR des FPIs möglichst genau zu kennen. Um
Frequenzabhängige optische Elemente zu eichen, wird immer eine bzw. mehrere
Referenzfrequnez(en) benötigt, die wiederum nur bis auf eine endliche
Genauigkeit bestimmt werden kann/können. Um das FPI zu eichen können
verschiedene Methoden angewandt werden.\par
Eine in der Arbeit \cite{kuschnick:2000:diplomarbeit} vorgestellte Methode 
bedient sich Hyperfeinstrukturübergänge des Cäsiumatoms, für die sehr genaue
Literaturwerte mit Fehlern von wenigen KHz \cite{PhysRevA.38.1616} bekannt sind
und als absolute Relativfrequenzen dienen können. Der damals gemessene FSR wurde
mit einer Genauigkeit von $6,7\cdot10^{-6}$ bzw. $4,0\cdot10^{-5}$ bestimmt. Da
diese Methode eines hohen experimentellen Aufwands bedarf, wurde hier
eine andere sog. \textit{Nonius-Methode} angewandt, welche bereits im Rahmen der
Arbeit \cite{schumann:2005:dissertation} zum Einsatz kam. Hierbei werden eine
Reihe von Absolutfrequenzen eines Lasers mit dem Wavemeter gemessen, für die das
FPI transmittiv ist. Schnell zu realisieren ist dies durch Verfahren des
Laser-Fringes auf den He:Ne-Fringe. Die Relativfrequenzen sind also ganzzahlige
Vielfache eines FSRs. Idee ist es nun, den Wert für den FSR zu finden, der alle
Relativfrequenzen möglichst ganzzahlig teilt. Einfach zu finden ist dieser Wert durch Minimieren der Fehlerfunktion
\begin{equation}\label{eq:nonius}
	f(\text{FSR})=\sum\limits_i\left[\frac{\delta\nu_i}{\text{FSR}}-\text{Round}\left(\frac{\delta\nu_i}{\text{FSR}}\right)\right]^2\,.
\end{equation}
Ein Summand besteht aus eine Aneinanderreihung von Parabeln, deren Minima bei
$\nicefrac{\text{FSR}}{n}$ mit $n\in\N$ (siehe Abb.
\ref{fig:nonius_beispiel_01}).
%TODO: !Bild Nonius Beispiel 01
Es ist wichtig, sowohl große als auch kleine Relativfreuqenzen zu messen, damit sowohl in kleinen als auch in großen
Frequenzbereichen die Mehrdeutigkeiten eliminiert werden, was sich durch
Vergrößern der Relativfrequenz mit einem konstanten Faktor erreichen lässt:
\begin{equation}\label{eq:nonius_faktor}
	\nu_i=a\cdot\nu_{i-1}\,.
\end{equation}
Weiterhin ist darauf zu achten, dass dieser Faktor maximal in der Größenordnung
des Verhältnisses
\begin{equation}\label{eq:nonius_faktor}
	a\stackrel{!}{<}\frac{\text{FSR}}{2\Delta\nu}\,,
\end{equation}
liegt, wobei $\Delta\nu$ der Fehler des Wavemeters ($40\,$MHz) ist, da sonst
mehrere Minima von Summanden höherer Relativfrequenzen in einem Minimum
eines Summanden einer kleineren Relativfrequenz liegen. Das würde eine
Eindeutige Aussage über den wahren FSR erschweren
(siehe\ref{fig:nonius_beispiel_02}).
%TODO: !Bild Nonius Beispiel 02
Da bei einem Faktor $a=3$ große Probleme bei der eindeutigen Bestimmung des FSR
aufgetreten sind, wurden einige Simulationen mit \textit{Mathematica}
durchgeführt, die bei der Optimierung der Messparameter geholfen haben.
Abbildung \ref{nonius_simulationen} zeigt Simulationen für einige
Relativfrequenzen verschiedene Werte von $a$. Wie man sieht, eignet sich $a=2$
gut zur eindeutigen FSR-Bestimmung.\par
%TODO: !Bild Nonius Simulation
Dafür wurden die in Tab. \ref{tab:nonius} aufgelisteten Frequenzen mit dem
Diodenlaser \textit{TA-Pro} von \textit{Toptica} gemessen, welcher sich
von $745\,$nm bis $795\,$nm sehr mühelos in der Frequenz durchstimmen lässt.
Abbildung \ref{fig:nonius_messung} zeigt die Fehlerfunktion für die gemessenen
Frequenzen, wobei alle möglichen $136$ Relativfrequnzen berechnet wurden.
%TODO: !Bild Nonius Messung
