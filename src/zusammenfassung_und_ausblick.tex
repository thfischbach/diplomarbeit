Im Rahmen dieser Arbeit wurde eine Frequenzstabilisierung für das
umfangreiche Diodenlasersystem zur HR-RIMS an Uranisotopen entwickelt und
getestet. Dabei wurde auf die bewährte Stabilisierungstechnik \textit{Fringe-Offset-Locking} zurückgegriffen, welche
durch eine weitere kommerzielle interferometrische Technik (\textit{iScan})
ergänzt wurde, um einen Zugewinn an Geschwindigkeit und Sicherheit der Stabilisierung bzw. der
Frequenzverstimmungen zu erhalten.
Ein Teil der digitalen Datenverarbeitung der Stabilisierung wurde dabei auf
Mikrocontroller ausgelagert, um das deterministische Verhalten des Systems zu steigern. Zur
Experimentsteuerung, Laserkontrolle und Datenaufnahme wurde ein
\textit{Labview}-Programm entwickelt, mit dem Messungen
automatisch und ohne großen Aufwand durchgeführt werden können.\par
Damit die Frequenzkontrolle korrekte Werte liefert und die Laser auf die
richtigen Frequenzen stabilisiert werden können, musste der freie
Spektralbereich des Fabry-Perot-Interferometers neu vermessen werden. Dieser
konnte mit einer Genauigkeit von $8,7\cdot10^{-6}$ auf $(298,0856\pm0,0026)$MHz
bestimmt werden.\par
Weiterhin wurde die Frequenzstabilität der Laser beider
Systeme in den verschiedenen Modi \textit{freilaufend}, \textit{iScan-stabilisiert} und
\textit{iScan+FOL-stabilisiert} überprüft, wobei vormals, im alten System, nur
die FOL-Technik zu Einsatz kam. Die Kombination beider Techniken ermöglicht die
völlige Eliminierung von Frequenzdrifts bei gleichzeitiger Reduzierung des
Jitters auf $<5\,$MHz für den blauen Laser und $<1,5\,$MHz für die beiden roten
Laser, während der Jitter der Laser im alten System bei 
$<15\,$MHz liegt. Eine zusätzliche Messung von Schwebungsfrequenzen der beiden
roten Laser sowohl am neuen als auch am alten Lasersystem liefern die
effektiven Linienbreiten der Laser. Im neuen System wurden diese zu
$(4,38\pm0,16)\,$MHz mit \textit{iScan}-Stabilisierung und $(3,93\pm0,10)\,$MHz
mit \textit{iScan}+FOL-Stabilisierung bestimmt. Im FOL-stabilisierten alten
System wurde eine effektive Linienbreite der roten Laser von
$(10,63\pm0,61)\,$MHz gemessen.\par
Softwareseitig wurde eine neue Frequenzverstimmungsstrategie, die die
Vorteile der \textit{iScans} (Geschwindigkeit, Kurzzeitstabilität und Frequenzsicherheit)
nutzt, entwickelt und mit der FOL-Technik dergestalt kombiniert, dass große
Frequenzverstimmungen von mehreren GHz schnell (innerhalb weniger Sekunden) und sicher durchgeführt
werden können. Dabei wurde festgestellt, dass noch bestehende
Geschwindigkeitseinbußen maßgeblich durch die Kommunikation zwischen
\textit{iScan} und PC bestehen. Softwareseitige Optimierungen können daher
in Zukunft einen enormen Zugewinn an Geschwindigkeit erzielen.\par
Bis zum Abschluss dieser Arbeit konnten zur Charakterisierung des Systems eine
Reihe von spektroskopischen Messungen an $^{238}$U mit dem neuen System
durchgeführt werden. Dabei wurden verschiedene Anregungsschemata im atomaren
Spektrum des Urans getestet und bewertet. Im Zuge dieser Messungen konnten auch
weitere für die mehrstufige Resonanzionisation nutzbare Zwischenniveaus
gefunden werden, die zuvor fälschlicherweise als AI-Zustände klassifiziert
waren. Aktuell werden auf den vermessenen Übergängen Effizienz- und
Spektroskopiemessungen fortgeführt, mit dem Ziel ein effizientes und selektives Anregungsschema zu finden, um quantitativ aussagekräftige Analytik betreiben zu können. Alle zukünftigen Ergebnisse sollen in
\cite{hakimi:2012:dissertation} vorgestellt werden.\par
Im Anschluss an diese Arbeit sollen zusätzliche Erweiterungen der Software
implementiert werden. Dazu gehören Isotopenverhältnismessungen und Funktionen
wie das Ansteuern des Quadrupols und des Ofens sowie das Monitoring einer Reihe
von Temperatursensoren. Außerdem wird angestrebt die Software in Hinblick auf
zukünftige Erweiterungen modularer umzugestalten.\par
Mit dieser Arbeit wurde somit eine zukunftsorientierte Grundlage für
HR-RIMS-Untersuchungen an Uranisotopen geschaffen, wobei das System prinzipiell
auf andere Elemente und Anwendungsbereiche, die den Einsatz stabilisierter und
scanbarer Diodenlaser vorraussetzen, übertragbar ist.