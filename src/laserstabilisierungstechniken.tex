In der hochaufgelösten Resonanz-Ionisations-Spektroskopie ist es
sehr wichtig, über längere Messzeiten möglichst konstante Rahmenbedingungen für die Messung zu
schaffen. Auch die Reproduzierbarkeit von Messungen ist nur gewährleistet, wenn
die Experimentparameter konstant bleiben. Ein wesentlicher, wenn nicht der
wichtigste, Parametersatz des Experiments sind die Lasereigenschaften. Für
eine stabile Laserfrequenz ist die Stabilität des Resonators ausschlaggebend.
Vibrationen, Luftdichte- bzw. Verstärungsmediumsdichtefluktuationen und
Brechungsindexänderungen, hervorgerufen durch Temperaturschwankungen, sind für
Instabilitäten verantwortlich. Schon bei Schwankungen oder Drifts von wenigen
MHz kann es bei sehr schmalbandigen Atomaren Übergängen zu erheblichen
Fluktuationen in der Ionisatonsrate kommen. Daher ist eine aktive
Frequenzstabilisierung der Diodenlaser unerlässlich.\par
In diesem Kapitel sollen zunächst der Vollständigkeit halber bewährte
Frequenzstabilisierungstechniken wie
\textit{Hänsch-Couillaud} (Kap.
\ref{sec:haensch-couillaud}) und \textit{Pound-Drever-Hall} (Kap.
\ref{sec:pound-drever-hall}) vorgestellt
(\cite{noertershaeuser:physik_des_lasers}) und anschließend auf die in diesem
Projekt verwendete \textit{Fringe-Offset-Stabilisierung} (Kap.
\ref{sec:fringe-offset-stabilisierung}) näher eingegangen werden. Darüber hinaus
besteht auch die Anforderung, zwischen den Anregungsfrequenzen verschiedener Isotope möglichst schnell wechseln zu können. Dies ist allein mit der Fringe-Offset-Stabilisierung nur bedingt gegeben. Deshalb wurde diese Technik mit einer kommerziellen Stabilisierunugstechnik kombiniert, auf die und deren Kombination mit der Fringe-Offset-Technik am Ende dieses Kapitals eingegangen wird.\par
Um einen Laser auf einer Frequenz festzuhalten, benötigt
man eine Referenz. Diese kann ein passiver, stabiler Resonator, ein atomarer
bzw. molekularer Übergang oder ein absolut stabiler Laser wie z.B. ein Helium-Neon-Laser sein.
Aus der Abweichung der Frequenz des zu stabilisierenden Lasers $\nu_{ist}$ zur
Sollfrequenz $\nu_{soll}$ muss ein Fehlersignal
\begin{equation}\label{eq:servoschleife_fehlersignal}
	S\approx C\cdot(\nu_{ist}-\nu_{soll})=C\cdot\delta\nu
\end{equation}
erzeugt werden, das hier exemplarisch linear zur Frequenzdifferenz $\delta\nu$
ist. Dieses Fehlersignal dient als Eingangssignal für eine Regelschleife, die
die Frequenz des Lasers auf die Sollfrequenz regelt. Auf Details dieser
Regelschleife soll später genauer eingegangen werden (siehe
\ref{sec:regeltechnik}). Die Generierung des Fehlersignals kann auf verschiedene Arten geschehen, wie im Folgenden zu sehen
ist.

\section{Hänsch-Couillaud}\label{sec:haensch-couillaud}
Die Frequenzstabilisierung nach Hänsch und Couillaud
basiert auf der \textit{Polarisationsspektroskopie}. Mit Hilfe eines
optischen Elemtents in einem festen Resonators als Referenz können verschiedene
Polarisationen des Laserlichts verschiedene Verluste erfahren. Dies lässt sich
mit einem doppelbrechenden Kristall oder einer Glasscheibe im Brewsterwinkel
realisieren. Man kann das elektrische Feld des linear polarisierten einfallenden
Lichts $E_0$ in Komponenten senkrecht und parallel zur Polarisationsrichtung mit
minimalen Verlusten zerlegen:
\begin{equation}\label{eq:haensch-couillaud_01}
	\begin{split}
		E_{\perp}^{(0)} & = E_0\cdot\cos{(\Theta)}\\
		E_{\parallel}^{(0)} & = E_0\cdot\sin{(\Theta)}\,.
	\end{split}
\end{equation}
Dabei ist $\Theta$ der Winkel zwischen einfallender Polarisation und
Polarisation mit maximalen bzw. minimalen Verlusten. $E_{\perp}^{(0)}$ wird also
im Wesentlichen vom Einkoppelspiegel reflektiert. $E_{\parallel}^{(0)}$
wird hingegen in den Resonator eingekoppelt und erfärt bei Nicht-Resonanz im
Resonator eine Phasenverschiebung $\delta$ zu $E_{\perp}^{(0)}$. Im Resonanzfall
ist diese Phasenverschiebung null. Durch Kenntnis der Phasenverschiebung kann
also eine Aussage über die Verstimmung zur Resonanz getroffen werden. Die
Komponenten des refelktierten Lichts
\begin{equation}\label{eq:haensch-couillaud_02}
	\begin{split}
		E_{\perp}^{(r)} & = E_{\perp}^{(0)}\cdot r_1\\
		E_{\parallel}^{(r)} & = E_{\parallel}^{(0)}\cdot\left(r_1-\frac{t_1^2r\mathrm{e}^{-\mathrm{i}\delta}}{r_1\left(1-r\mathrm{e}^{-\mathrm{i}\delta}\right)}\right)
	\end{split}
\end{equation}
ergeben zusammen elliptisch polarisiertes Licht und somit eine Überlagerung von
$\sigma^+$- und $\sigma^-$-Licht mit unterschiedlichen Amplituden. Dabei sind
$r_1$ und $t_1$ Reflexions- und Transmissionskoeffizienten des
Einkoppelspiegels. $r$ beschreibt die Verluste durch die Umläufe im Resonator.
Bei Phasenverschiebung überwiegt einmal der $\sigma^+$-Anteil und einmal der $\sigma^-$-Anteil. Bei Resonanz sind beide Anteile gleich und es entsteht wieder linear polarisiertes Licht. Eine $\nicefrac{\lambda}{4}$-Platte erzeugt aus
beiden zirkularen Anteilen ($\sigma^+$ und $\sigma^-$) lineare Anteile, welche
durch einen Polarisationsstrahlteiler getrennt und letzendlich durch Photodioden
detektiert werden können. Abbildung \ref{fig:haensch-couillaud} zeigt den
optischen Aufbau. Die Differenz der zu den Feldstärken der elektrischen Komponente
des Lichts proportionalen Ströme der Photodioden
\begin{equation}\label{eq:servoschleife_fehlersignal}
	(I_1-I_2)\propto\abs{E^{(0)}}^2\cos{(\Theta)}\sin{(\Theta)}\frac{t_1^2r^2\sin{(\delta)}}{(1-r^2)+4r^2\sin^2{\left(\frac{\delta}{2}\right)}}
\end{equation}
lieftert das in Abb. \ref{fig:haensch-couillaud_fehlersignal} gezeichnete
Fehlersignal mit Nulldurchgang bei den Resonanzen $\delta=2\pi n$. Die
Regelschleife muss also auf diesen Nulldurchgang regeln.

\section{Pound-Drever-Hall}\label{sec:pound-drever-hall}

\section{Fringe-Offset-Stabilisierung}\label{sec:fringe-offset-stabilisierung}

