In der hochaufgelösten Resonanz-Ionisations-Spektroskopie ist es
sehr wichtig, über längere Messzeiten möglichst konstante Rahmenbedingungen für die Messung zu
schaffen. Auch die Reproduzierbarkeit von Messungen ist nur gewährleistet, wenn
die Experimentparameter konstant bleiben. Ein wesentlicher, wenn nicht der
wichtigste, Parametersatz des Experiments sind die Lasereigenschaften. Für
eine stabile Laserfrequenz ist die Stabilität des Resonators ausschlaggebend.
Vibrationen, Luftdichte- bzw. Verstärungsmediumsdichtefluktuationen und
Brechungsindexänderungen, hervorgerufen durch Temperaturschwankungen, sind für
Instabilitäten verantwortlich. Schon bei Schwankungen oder Drifts von wenigen
MHz kann es bei sehr schmalbandigen Atomaren Übergängen zu erheblichen
Fluktuationen in der Ionisatonsrate kommen. Daher ist eine aktive
Frequenzstabilisierung der Diodenlaser unerlässlich.\par
In diesem Kapitel sollen zunächst bewährte Frequenzstabilisierungstechniken
vorgestellt (\cite{noertershaeuser:physik_des_lasers}) und anschließend auf die
in diesem Projekt verwendete \textit{Fringe-Offset-Stabilisierung}
(Kap. \ref{sec:fringe-offset}) näher eingegangen werden. Darüber hinaus besteht
auch die Anforderung, zwischen den Anregungsfrequenzen verschiedener Isotope möglichst schnell wechseln zu können. Dies ist allein mit der
Fringe-Offset-Stabilisierung nur bedingt gegeben. Deshalb wurde diese
Technik mit einer kommerziellen Methode kombiniert, auf die am Ende dieses
Kapitals eingegangen wird.

\section{}