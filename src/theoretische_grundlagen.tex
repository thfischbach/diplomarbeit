Im folgenden Kapitel sollen die wichtigsten physikalischen Grundlagen für das
Verständnis der in dieser Arbeit behandelten Thematik erklärt werden. Dabei
werden die Kenntnisse über den Aufbau des Atoms, die quantisierten Lösungen der
Schrödinger- bzw. Diracgleichung, die dabei auftretenden Drehimpulskopplungen
und die daraus resultierenden Energieaufspaltungen der Atome als vorhanden
vorausgesetzt. Diesbezüglich sei auf einschlägige Lehrbücher wie
\cite{demtroeder:ex3} verwiesen.\\
In Kapitel \ref{sec:licht-atom-wechselwirkung} soll die Wechselwirkung von
Licht und Atom behandelt werden. Darauf aufbauend soll
in Kapitel \ref{sec:ris} die Resonanz-Ionisations-Spektroskopie,
kurz RIS, in Bezug auf die Resonanz-Ionisation von Uran-Isotopen erklärt
werden. Kapitel \ref{sec:diodenlaser} soll das Funktionsprinzip von
Halbleiterlasern und deren Rolle in diesem Projekt wiedergeben.

\section{Lich-Atom-Wechselwirkung}\label{sec:licht-atom-wechselwirkung}

Essenziell wichtig für die Resonanz-Ionisation ist das Verständnis der atomaren
Wechselwirkung mit einem Lichtfeld. Im Folgenden soll insbesondere auf die
atomaren Übergänge und deren Linienprofil eingegangen werden, da dieses bei der
Atom-Spektroskopie eine wichtige Rolle spielt.


%$$P_{ik}=\frac{2\pi}{\hbar^2}\lvert\langle\psi_k^0\rvert\hat{\mathcal{H}}'\lvert\psi_i^0\rangle\rvert^2\delta(E_k^0-E_i^0+\hbar\omega)$$
%$$W_{ki}=\frac{\pie^2}{3\varepsilon_0\hbar^2}\lvert\langle\psi_k\rvert\vec{r}\lvert\psi_i\rangle\rvert^2\cdot\omega_{\nu}$$
%$$B_{ki}=\frac{2}{3}\frac{\pi^2e^2}{\varepsilon_0\hbar^2}\lvert\langle\psi_k\rvert\vec{r}\lvert\psi_i\rangle\rvert^2$$


\subsection{Übergangsraten}\label{subsec:uebergangsraten}
Befindet sich ein Atom in einem Lichtfeld können Absorptions- und
Emmissions-Prozesse beobachtet werden. Im ersten Fall absorbiert das Atom ein
Photon einer bestimmten Mode mit der Energie $\hbar\omega_L$ aus dem Lichtfeld.
Dabei wird das Atom von einem Zustand in einen entsprechend der Photonenenergie
energetisch höher gelegenen Zustand überführt:
\begin{equation}\label{eq:uebergang}
	E_f-E_i=\hbar\omega_L \, .
\end{equation}
Dabei ist zu beachten, dass gebundene Zustände immer quantisiert sind, also
nicht kontinuierlich im Energiespektrum verteilt sind. Analog kann ein Atom ein
Photon in das Lichtfeld emittieren. Die Zustandsänderung des Atoms folgt dann entsprechend von einem Zustand in ein energetisch niedriger gelegenen
Zustand.\\
Grundlegend beschreibt \textit{Fermis goldene Regel} die
Übergangswahrscheinlichkeit von einem Zustand $\Psi_i$ in einen beliebigen
Zustand $\Psi_f$. Diese kann halbklassisch störungstheoretisch hergeleitet
werden. Dabei betrachtet man das externe Lichtfeld als Störung $\OPH'(t)$
zusätzlich zum zeitunabhängigen Hamilton-Operator $\OPH_0$.
\begin{equation}\label{eq:hamilton_stoerung}
	\OPH(t)=\OPH_0+\OPH'(t)
	\quad\text{mit}\quad
	\OPH'(t)=\E_0 \mathrm{e}^{\mathrm{i}(\vk\cdot\vr-\omega_L t)}\approx\E_0
	\mathrm{e}^{-\mathrm{i}\omega_L t}
\end{equation}
Hierbei wurde die berechtigte Annahme gemacht, dass das Atom in Relation zur
Wellenlänge des Lichtfelds sehr klein ist und es somit am Ort des Atoms nur
verschwindend geringe örtliche Änderungen der Amplitude des Lichtfelds gibt
($\vk\cdot\vr\ll 1$, Atom am Ort $\vr=0$). Diese Näherung nennt man
\textit{Dipolnäherung}. Nun setzt man mit der zeitabhängigen Schrödingergleichung an und entwickelt
$\Psi(t)$ in die stationären Eigenzustände $\Psi_n$ des Atoms:
\begin{equation}\label{eq:sgl_stoerung_01}
	\mathrm{i}\hbar\pfrac{}{t}\ket{\Psi(t)}=\OPH(t)\ket{\Psi(t)}\,,
	\quad
	\ket{\Psi(t)}=\sum_n{c_n(t)\mathrm{e}^{-\frac{\mathrm{i}}{\hbar} E_n
	t}\ket{\Psi_n}}\,.
\end{equation}
Führt man die Zeitableitung aus und projeziert beide Seiten der Gleichung auf
einen beliebigen Zustand $\Psi_k$, findet man
\begin{equation}\label{eq:sgl_stoerung_02}
	\dot
	c_k(t)=-\frac{\mathrm{i}}{\hbar}\sum_n{c_n(t)\mathrm{e}^{-\mathrm{i}\omega_{nk}}\bra{\Psi_k}\OPH'(t)\ket{\Psi_n}}
	\quad\text{mit}\quad
	\omega_{nk}=\frac{E_k-E_n}{\hbar}\,.
\end{equation}
Nimmt man nun an, dass die Störung klein ist, das Atom also vornehmlich im
Ausgangs-Zustand $\Psi_i$ bleibt ($c_i(0)=1$, $c_i(t)\approx 1$), verkürzt
sich die Summe in Gleichung \ref{eq:sgl_stoerung_02} auf einen Summanden mit $n=i$.
Für die Übergangswahrscheinlichkeit ergibt sich mit Gleichung \ref{eq:hamilton_stoerung} nach
zeitlicher Integration des Koeffizienten des Endzustandes $\dot c_f(t)$
\begin{eqnarray}\label{eq:uebergangs_wkt}
	\begin{split}
		P_{i\to f}(t,\Delta\omega) &= \abs{\int_0^t{\dot c_f(t')\dd t'}}^2\\
		& =
		\frac{1}{\hbar^2}\abs{\bra{\Psi_f}\OPH'\ket{\Psi_i}}^2\cdot\frac{\sin^2{\left(\frac{\Delta\omega}{2}t\right)}}{\left(\frac{\Delta\omega}{2}\right)^2}\\
		&\text{mit}\quad\Delta\omega=\omega_L-\omega_{fi}\,,\quad\omega_{fi}=-\omega_{if}\,.
	\end{split}
\end{eqnarray}
Für die totale Übergangsrate gilt dann nach genügend langer Zeit
\begin{equation}\label{eq:uebergangs_rate}
	\Gamma_{i\to f} =
	\lim_{t\to\infty}{\left(\fracd{}{t}{\left(\int_{-\infty}^{\infty}{P_{i\to
	f}(t)\rho(\omega_L)}\dd\omega\right)}\right)}\,.
\end{equation}


\subsection{Auswahlregeln}\label{subsec:auswahlregeln}

%\subsubsection{Einstein-Koeffizienten}\label{subsec:einstein-koeffizienten}

\subsection{Linienprofil}\label{sec:linienprofil}

\section{Resonanz-Ionisations-Spektroskopie}\label{sec:ris}

\section{Diodenlaser in der RIS}\label{sec:diodenlaser}
%\cite{chow:semiconductor-laser}