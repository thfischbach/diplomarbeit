Im folgenden Kapitel sollen die theoretischen Grundlagen für das Verständnis der
in dieser Arbeit behandelten Thematik erklärt werden. Dabei werden die
Kenntnisse über den Aufbau des Atoms, die quantisierten Lösungen der
Schrödinger- bzw. Diracgleichung, die auftretenden Drehimpulskopplungen und die
daraus resultierenden Energieaufspaltungen der Atome als vorhanden
vorausgesetzt. Diesbezüglich sei auf einschlägige Lehrbücher wie
\cite{demtroeder:ex3} verwiesen.\\
In \ref{subsec:licht-atom-wechselwirkung} soll die Wechselwirkung von Licht
und Atom anhand eines 2-Niveau-Systems veranschaulicht werden. Darauf aufbauend soll
in \ref{subsec:ris} die Resonanz-Ionisations-Spektroskopie,
kurz RIS, in Bezug auf die Resonanz-Ionisation von Uran-Isotopen erklärt
werden. \ref{subsec:diodenlaser} soll das Funktionsprinzip von
Halbleiterlasern und deren Rolle in diesem Projekt wiedergeben.

\subsection{Lich-Atom-Wechselwirkung}\label{subsec:licht-atom-wechselwirkung}



\subsubsection{Übergangsraten}\label{subsec:uebergangsraten}

\subsubsection{Auswahlregeln}\label{subsec:auswahlregeln}

\subsubsection{Einstein-Koeffizienten}\label{subsec:einstein-koeffizienten}

\subsubsection{Linienprofil}\label{subsec:linienprofil}

\subsection{Resonanz-Ionisations-Spektroskopie}\label{subsec:ris}

\subsection{Diodenlaser in der RIS}\label{subsec:diodenlaser}
%\cite{chow:semiconductor-laser}