Im folgenden Kapitel sollen die wichtigsten physikalischen Grundlagen für das
Verständnis der in dieser Arbeit behandelten Thematik erklärt werden. Dabei
werden die Kenntnisse über den Aufbau des Atoms, die quantisierten Lösungen der
Schrödinger- bzw. Diracgleichung, die dabei auftretenden Drehimpulskopplungen
und die daraus resultierenden Energieaufspaltungen der Atome als vorhanden
vorausgesetzt. Diesbezüglich sei auf einschlägige Lehrbücher wie
\cite{demtroeder:ex3} verwiesen.\\
In Kapitel \ref{sec:licht-atom-wechselwirkung} soll die Wechselwirkung von
Licht und Atom behandelt werden. Darauf aufbauend soll
in Kapitel \ref{sec:ris} die Resonanz-Ionisations-Spektroskopie,
kurz RIS, in Bezug auf die Resonanz-Ionisation von Uran-Isotopen erklärt
werden. Kapitel \ref{sec:diodenlaser} soll das Funktionsprinzip von
Halbleiterlasern und deren Rolle in diesem Projekt wiedergeben.

\section{Lich-Atom-Wechselwirkung}\label{sec:licht-atom-wechselwirkung}

Essenziell wichtig für die Resonanz-Ionisation ist das Verständnis der atomaren
Wechselwirkung mit einem Lichtfeld. Im Folgenden soll insbesondere auf die
atomaren Übergänge und deren Linienprofil eingegangen werden, da dies bei der
Atom-Spektroskopie eine wichtige Rolle spielt.


% $$P_{ik}=\frac{2\pi}{\hbar^2}\lvert\langle\psi_k^0\rvert\hat{\mathcal{H}}'\lvert\psi_i^0\rangle\rvert^2\delta(E_k^0-E_i^0+\hbar\omega)$$
% $$W_{ki}=\frac{\pi e^2}{3\varepsilon_0\hbar^2}\lvert\langle\psi_k\rvert\vec{r}\lvert\psi_i\rangle\rvert^2\cdot\omega_{\nu}$$
% $$B_{ki}=\frac{2}{3}\frac{\pi^2e^2}{\varepsilon_0\hbar^2}\lvert\langle\psi_k\rvert\vec{r}\lvert\psi_i\rangle\rvert^2$$
% $$\vec{M}_{ik}=e\langle\psi_i\rvert\vec{r}\lvert\psi_k\rangle$$


\subsection{Übergangsraten}\label{subsec:uebergangsraten}
Befindet sich ein Atom in einem Lichtfeld können Absorptions- und
Emissions-Prozesse beobachtet werden. Im ersten Fall absorbiert das Atom ein
Photon einer bestimmten Mode mit der Energie $\hbar\omega_L$ aus dem Lichtfeld.
Dabei wird das Atom von einem Zustand in einen entsprechend der Photonenenergie
energetisch höher gelegenen Zustand überführt:
\begin{equation}\label{eq:uebergang}
	E_f-E_i=\hbar\omega_L \, .
\end{equation}
Dabei ist zu beachten, dass gebundene Zustände immer quantisiert sind, also
nicht kontinuierlich im Energiespektrum verteilt sind. Analog kann ein Atom ein
Photon in das Lichtfeld emittieren. Die Zustandsänderung des Atoms folgt dann entsprechend von einem Zustand in ein energetisch niedriger gelegenen
Zustand.\\
Grundlegend beschreibt \textit{Fermis goldene Regel} die
Übergangsrate von einem Zustand $\Psi_i$ (i: \textit{initial}) in einen
beliebigen Zustand $\Psi_f$ (f: \textit{final}). Diese kann halbklassisch
störungstheoretisch hergeleitet werden.  Dabei betrachtet man das externe Lichtfeld als Störung
zusätzlich zum zeitunabhängigen Hamilton-Operator $\OPH_0$:
\begin{equation}\label{eq:hamilton}
	\OPH(t)=\OPH_0+\OPH'(t)
\end{equation}
mit dem Wechselwirkungsoperator
\begin{equation}\label{eq:ww}
	\begin{split}
		\OPH'(t) &= -E_0\vec{\epsilon}\cdot\OPv{d}\cos{(\vk\cdot\vr-\omega_L t)}\\
		&=
		-\OPH'\left(\mathrm{e}^{\mathrm{i}(\vk\cdot\vr-\omega_L
		t)}+\mathrm{e}^{-\mathrm{i}(\vk\cdot\vr-\omega_L t)}\right)\\
		&\text{mit}\quad
		\OPH'=\frac{1}{2}E_0\vec{\epsilon}\cdot\OPv{d}\,.
	\end{split}
\end{equation}
Hierbei ist $E_0$ die Amplitude und $\vec{\epsilon}$ der
Polarisierungsvektor des Lichtfelds. $\OPv{d} = e\OPv{r}$ ist der Dipoloperator. 
Zusätzlich kann man die Annahme machen, dass das Atom in Relation zur
Wellenlänge des Lichtfelds sehr klein ist und es somit am Ort des Atoms nur
verschwindend geringe örtliche Änderungen der Amplitude des Lichtfelds erfährt
($\vk\cdot\vr\ll 1$, Atom am Ort $\vr=0$). Diese Näherung nennt man
\textit{Dipolnäherung}. Dadurch folgt für die Entwicklung des
Ortsteils der Exponentialfunktionen in erster Ordnung
$\mathrm{e}^{\pm\mathrm{i}\vk\cdot\vr}\approx1$. Nun setzt man mit der zeitabhängigen Schrödingergleichung an und entwickelt $\Psi(t)$ in die stationären Eigenzustände $\Psi_n$ des Atoms:
\begin{equation}\label{eq:sgl_stoerung_01}
	\mathrm{i}\hbar\pfrac{}{t}\ket{\Psi(t)}=\OPH(t)\ket{\Psi(t)}\,,
	\quad
	\ket{\Psi(t)}=\sum_n{c_n(t)\mathrm{e}^{-\frac{\mathrm{i}}{\hbar} E_n
	t}\ket{\Psi_n}}\,.
\end{equation}
Führt man die Zeitableitung aus und projiziert beide Seiten der Gleichung auf
einen beliebigen Zustand $\Psi_f$, findet man
\begin{equation}\label{eq:sgl_stoerung_02}
	\dot
	c_f(t)=-\frac{\mathrm{i}}{\hbar}\sum_n{c_n(t)\mathrm{e}^{-\mathrm{i}\omega_{nf}t}\bra{\Psi_f}\OPH'(t)\ket{\Psi_n}}
	\quad\text{mit}\quad
	\omega_{nf}=\frac{E_n-E_f}{\hbar}\,.
\end{equation}
Nimmt man nun an, dass die Störung klein ist, das Atom also vornehmlich im
Ausgangs-Zustand $\Psi_i$ bleibt ($c_i(0)=1$, $c_i(t)\approx 1$, $c_{n\neq
i}\approx0$), verkürzt sich die Summe in Gleichung \eqref{eq:sgl_stoerung_02}
auf einen Summanden mit $n=i\neq f$. $c_f(t)$ folgt aus zeitlicher Integration von $\dot c_f(t)$ mit
Gleichung \eqref{eq:ww}:
\begin{equation}\label{eq:koeff_cf}
	\begin{split}
		c_f(t) &= \int_0^t{\dot c_f(t')\dd t'}\\
		&=
		-\frac{\mathrm{i}}{\hbar}\bra{\Psi_f}\OPH'\ket{\Psi_i}\left[\frac{\mathrm{e}^{\mathrm{i}(\omega_{fi}-\omega_L)t}-1}{\mathrm{i}(\omega_{fi}-\omega_L)}+\frac{\mathrm{e}^{\mathrm{i}(\omega_{fi}+\omega_L)t}-1}{\mathrm{i}(\omega_{fi}+\omega_L)}\right]\\
		&\text{mit}\quad
		\omega_{fi}=-\omega_{if}
	\end{split}
\end{equation}
Zu beachten ist hierbei, dass im Falle der Absorption der Nenner des zweiten
Bruchs in Gleichung \eqref{eq:koeff_cf} im nahresonenten Fall
($\omega_{fi}\approx\omega_L$) in der Größenordnung $2\omega_{fi}\approx
\unit{10^{15}}{s^{-1}}$ (bei optischen Übergängen) liegt. Somit wird der
zweite Bruch verschwindend gering gegenüber dem ersten und kann vernachlässigt
werden. Im Falle der Emission betrachtet man Übergänge vom energetisch höheren
Niveau zum energetisch niedrigeren Niveau. Dabei gilt
($\omega_{if}\approx\omega_L$), wodurch der erste Bruch vernachlässigt werden
kann. Im Folgenden wird allerdings exemplarisch die Absorption betrachtet.\\
Für die Übergangswahrscheinlichkeit ergibt sich
\begin{equation}\label{eq:uebergangs_wkt}
	\begin{split}
		P_{i\to f}(t,\Delta\omega) &= \abs{c_f(t)}^2\\
		& =
		\frac{1}{\hbar^2}\abs{\bra{\Psi_f}\OPH'\ket{\Psi_i}}^2\cdot\frac{\sin^2{\left(\frac{\Delta\omega}{2}t\right)}}{\left(\frac{\Delta\omega}{2}\right)^2}\\
		&\text{mit}\quad
		\Delta\omega=\omega_L-\omega_{fi}\,.
	\end{split}
\end{equation}
Für die totale Übergangswahrscheinlichkeit gilt dann mit Gleichung
\eqref{eq:uebergangs_wkt}
\begin{equation}\label{eq:uebergangs_wkt_total}
	\begin{split}
		P_{i\to f}(t)
		&= \int{P_{i\to f}(t,\Delta\omega)\rho(E_f)}\dd E_f\\
		&\approx \rho(E_f)\int{P_{i\to f}(t,\Delta\omega)}\dd E_f\\
		&= \hbar\rho(E_f)\int_{-\infty}^{\infty}{P_{i\to
		f}(t,\Delta\omega)}\dd(\Delta\omega)\\
		&= \frac{2\pi t}{\hbar}\rho(E_f)\abs{\bra{\Psi_f}\OPH'\ket{\Psi_i}}^2\,.
	\end{split}
\end{equation}
Hierbei wurde die Energieniveaudichte $\rho(E_f)=\fracd{n}{E_f}$ eingeführt. Sie
beschreibt die Verteilung der Energieniveaus der Endzustände $E_f$. Mit
der Annahme, dass sich $\rho(E_f)$ gegenüber $P_{i\to f}(t,\Delta\omega)$ nur langsam ändert, kann $\rho(E_f)$ in Gleichung \eqref{eq:uebergangs_wkt_total}
als hinreichend konstant angenommen und aus dem Integral gezogen werden.
Weiterhin wurde die Substitution $\dd E_f=\hbar\dd(\Delta\omega)$ vorgenommen. Für die
totale Übergangsrate folgt dann mit Gleichung \eqref{eq:uebergangs_wkt_total}
für genügend große Zeiten ($t\gg\frac{1}{\omega_{fi}}$) Fermis goldene Regel:
\begin{equation}\label{eq:uebergangs_rate_total}
	\begin{split}
		\Gamma_{i\to f}
		&= \lim_{t\to\infty}{\left(\fracd{}{t}{P_{i\to f}(t)}\right)}\\
		&= \frac{2\pi}{\hbar}\rho(E_f)\abs{\bra{\Psi_f}\OPH'\ket{\Psi_i}}^2\,.
	\end{split}
\end{equation}
% $\bra{\Psi_f}\OPH'\ket{\Psi_i} =
% E_0\vec{\epsilon}\cdot e\bra{\Psi_f}\OPv{r}\ket{\Psi_i}$ enthält das sog.
% \textit{Dipolmatrixelement} $e\bra{\Psi_f}\OPv{r}\ket{\Psi_i}$ für die Zustände
% $\Psi_i$ und $\Psi_f$. Es gibt an, ob ein Zustand im Rahmen der Dipolnäherung erlaubt oder verboten ist.
% Dazu sei auf Kapitel \ref{subsec:auswahlregeln} verwiesen.\par





\subsection{Auswahlregeln}\label{subsec:auswahlregeln}

%\subsubsection{Einstein-Koeffizienten}\label{subsec:einstein-koeffizienten}

\subsection{Linienprofil}\label{sec:linienprofil}

\section{Resonanz-Ionisations-Spektroskopie}\label{sec:ris}

\section{Diodenlaser in der RIS}\label{sec:diodenlaser}
%\cite{chow:semiconductor-laser}