Im folgenden Kapitel sollen die wichtigsten physikalischen Grundlagen für das
Verständnis der in dieser Arbeit behandelten Thematik erklärt werden. Dabei werden die
Kenntnisse über den Aufbau des Atoms, die quantisierten Lösungen der
Schrödinger- bzw. Diracgleichung, die dabei auftretenden Drehimpulskopplungen
und die daraus resultierenden Energieaufspaltungen der Atome als vorhanden
vorausgesetzt. Diesbezüglich sei auf einschlägige Lehrbücher wie
\cite{demtroeder:ex3} verwiesen.\\
In Kapitel \ref{subsec:licht-atom-wechselwirkung} soll die Wechselwirkung von
Licht und Atom behandelt werden. Darauf aufbauend soll
in Kapitel \ref{subsec:ris} die Resonanz-Ionisations-Spektroskopie,
kurz RIS, in Bezug auf die Resonanz-Ionisation von Uran-Isotopen erklärt
werden. Kapitel \ref{subsec:diodenlaser} soll das Funktionsprinzip von
Halbleiterlasern und deren Rolle in diesem Projekt wiedergeben.

\subsection{Lich-Atom-Wechselwirkung}\label{subsec:licht-atom-wechselwirkung}

Essenziell wichtig für die Resonanz-Ionisation ist das Verständnis der atomaren
Wechselwirkung mit einem Lichtfeld. Im Folgenden soll insbesondere auf die
atomaren Übergänge und deren Linienprofil eingegangen werden, da dieses bei der
Atom-Spektroskopie eine wichtige Rolle spielt.


%$$P_{ik}=\frac{2\pi}{\hbar^2}\lvert\langle\psi_k^0\rvert\hat{\mathcal{H}}'\lvert\psi_i^0\rangle\rvert^2\delta(E_k^0-E_i^0+\hbar\omega)$$
%$$W_{ki}=\frac{\pie^2}{3\varepsilon_0\hbar^2}\lvert\langle\psi_k\rvert\vec{r}\lvert\psi_i\rangle\rvert^2\cdot\omega_{\nu}$$
%$$B_{ki}=\frac{2}{3}\frac{\pi^2e^2}{\varepsilon_0\hbar^2}\lvert\langle\psi_k\rvert\vec{r}\lvert\psi_i\rangle\rvert^2$$


\subsubsection{Übergangsraten}\label{subsec:uebergangsraten}
Befindet sich ein Atom in einem Lichtfeld können Absorptions- und
Emmissions-Prozesse beobachtet werden. Im ersten Fall absorbiert das Atom ein
Photon einer bestimmten Mode aus dem Lichtfeld. Dabei wird das Atom von einem
Zustand ($$\bra{\Psi}$$) in einen entsprechend der Photonenenergie
$$\hbar\omega$$ energetisch höher gelegenen Zustand überführt. Analog kann ein Atom ein Photon in das Lichtfeld
emittieren. Die Zustandsänderung des Atoms folgt dann entspechend von einem
Zustand in ein energetisch niedriger gelegenen Zustand.\\


\subsubsection{Auswahlregeln}\label{subsec:auswahlregeln}

%\subsubsection{Einstein-Koeffizienten}\label{subsec:einstein-koeffizienten}

\subsubsection{Linienprofil}\label{subsec:linienprofil}

\subsection{Resonanz-Ionisations-Spektroskopie}\label{subsec:ris}

\subsection{Diodenlaser in der RIS}\label{subsec:diodenlaser}
%\cite{chow:semiconductor-laser}