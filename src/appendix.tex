\chapter{Elektronik der
Signalgenerierung}\label{anh:kap:signalgenerierung_elektronik} In diesem
Abschnitt soll die Elektronik der Signalgenerierung, die in Abschn. \ref{subsec:signalgenerierung} erwähnt wurde, genauer betrachtet werden.
Ziel ist es, aus dem Fringepattern-Signal des FPIs digital lesbare Signale zu
generieren, mit denen man die Zeit zwischen Beginn der
Spannungsrampe und Offsetfringe bzw. zwischen Offsetfringe
und Interfringe messen kann. Exemplarisch wird die Signalgenerierung hier
für einen Laser erklärt. Selbstverständlich besteht der komplette
Schaltkreis aus vier identischen Kopien der in Abb.
\ref{fig:signalgenerierung_schaltplan} dargestellten Schaltung.\par
\begin{figure}[h]
 	\centering
 	\fbox{\parbox{\dimexpr \linewidth - 2\fboxrule - 2\fboxsep}{
		\subfigure[]{
			\label{subfig:signalgenerierung_schaltplan_01}
	    	\includegraphics[width=(\textwidth-1cm)]{gfx/signalgenerierung_schaltplan_01}
	    }
		\subfigure[]{
			\label{subfig:signalgenerierung_schaltplan_02}
	    	\includegraphics[width=(\textwidth-1cm)]{gfx/signalgenerierung_schaltplan_02}
	    	}
	    \subfigure[]{
			\label{subfig:signalgenerierung_schaltplan_03}
	    	\includegraphics[width=(\textwidth-1cm)]{gfx/signalgenerierung_schaltplan_03}
	    	}
	}}
	\caption[Signalgenerierung
	Schaltplan]{Schaltplan der
	Signalgenerierung zum
	Fringe-Offset-Locking (aus ***blabla*Referenz zu
	Höltke Schaltplan*blabla***)}\label{fig:signalgenerierung_schaltplan}
\end{figure}
Um eletronisch ein Fringemaximum zu detektieren ist es nötig die Ableitung
des Signals zu generieren. Abbildung
\ref{subfig:signalgenerierung_schaltplan_01} zeigt den ersten Teil der Schaltung. Über den rückgekoppelten Operationsverstärker
\textit{TL082} wird die Ableitung gebildet und somit ein Signal an (7)
erzeugt, das einen steilen Nulldurchgang mit negativer Steigung bei jedem Fringe hat und über einen
Monitor-Ausgang (DIF0) betrachtet werden kann. Danach wird ein
hysteretischer Komparator \textit{AD790} verwendet, der oberhalb eines
bestimmten Spannungsniveaus ein digitales HIGH und unterhalb eines bestimmten Spannungsniveaus
ein digitales LOW an (7) ausgibt. Der Zwischenzustand von (7) ist
davon abhängig, ob die Spannung von unterhalb des unteren Niveaus oder von
oberhalb des oberen Niveaus kommt. Somit schaltet der Komparator kurz vor dem
Maximum des Fringes auf HIGH und bei unterem Niveau von $0\,$V genau am
Nulldurchgang der Ableitung, also am Maximum des Fringes, auf LOW. Die
Teilschaltung mit dem IC \textit{74LS123} reagiert auf die fallende Flanke des
TTL-Pulses des Komparators und erzeugt beginnend am
Nulldurchgang der Ableitung selbst einen kurzen TTL-Puls (T1) mit einer
Länge, die durch die Wahl des vorgeschalteten Widerstandes und des Kondensators abhängt.\par
Das Gate des Rampengenerators ((STARTTTL) in Abb.
\ref{subfig:signalgenerierung_schaltplan_02}) ist bei fallender Spannungsrampe
HIGH und bei steigender Spannungsrampe LOW. Im ersten Teil der Schaltung
\ref{subfig:signalgenerierung_schaltplan_02} bewirkt dies einen durch das
Drehpotentiometer in der Länge einstellbaren TTL-Puls am Anfang der steigenden
Spannungsrampe. Das Ende der steigenden Spannungsrampe wird ignoriert.
Mit einer Zeitverzögerung von der Länge des erzeugten TTL-Pulses wird dann im
zweiten Teil der Schaltung \ref{subfig:signalgenerierung_schaltplan_02} ein weiterer TTL-Puls erzeugt.
Dadurch wird eine Verzögerung des Starttriggers der steigenden Rampe bewirkt, um
den Start der Fringe-Detektion zu verschieben und so die zu Beginn der Rampe
auftretende Nichtlinearität zu ignorieren.\par
Schaltung \ref{subfig:signalgenerierung_schaltplan_03} ist nun für die Erzeugung
der in \ref{fig:FPI_signal-zeitverlauf}(d,e) dargestellten TTL-Pulse zuständig. Sie
besteht aus zwei NOR-Flipflops (links und rechts) und wiederum aus dem IC
\textit{74LS123} in der Mitte, welcher nun sensitiv auf eine steigende
Pulsflanke ist. Im Ausgangszustand liegt an den beiden Eingängen und am unteren
Ausgang des linken Flipflops LOW an. Der verzögerte kurze Start-TTL-Puls kommt
nun am Set-Eingang des linken Flipflops (CLK1) an und setzt den unteren
Ausgang, der den Offsetfringe-TTL-Puls durch (CLKB1) repräsentiert, auf HIGH.
Der Counter für die Offsetfringezeit beginnt zu zählen. Der Offsetfringe
erzeugt nun an (T1) einen kurzen TTL-Puls, welcher den linken Flipflop zurücksetzt. (CLKB1) wird also LOW und der Counter
für die die Offsetfringezeit stoppt. Der obere Ausgang des ersten Flipflops
wechselt somit auf HIGH, was einen kurzen TTL-Puls am Ausgang Q von
\textit{74LS123} auslöst. Dadurch wird der Ausgang des rechten FlipFlops (CLKA1) auf HIGH gesetzt
und der Counter für die Interfringezeit beginnt zu zählen. Damit durch den
kurzen Puls an (T1) nicht der verbotene Eingangszustand HIGH/HIGH am rechten
Flipflop anliegt, wird dieser durch ein effektives AND mit
$\overline{\text{Q}}$, das zu diesem Zeitpunkt kurz LOW ist, verknüpft. Der
Interfringe löst wieder einen TTL Puls an (T1) aus und setzt nun den rechten
Flipflop zurück, was auch den Counter für die Interfringezeit stoppt.\par
Mit dieser Schaltung können also die Transmissionssignale des FPIs in digitale
Signale für die Counterkarte umgesetzt werden.

\chapter{Counterkarte}\label{anh:kap:counterkarte}
